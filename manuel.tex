\documentclass[a4paper,10pt]{article}
\usepackage[utf8]{inputenc}
\usepackage[T1]{fontenc}
\usepackage[french]{babel}
\usepackage{lmodern} %pack de police
\usepackage{tikz}

\author{Meryll \bsc{Essig}\\ Tamara \bsc{Rocacher}}
\date{\today}
\title{Manuel d'utilisation}


\begin{document}

\vfill
\maketitle
\vfill
\section*{Utilisation du programme}
A la première ouverture de l'application,
la liste des taches a effectuer est vide
et l'utilisateur peut commencer a saisir ses taches.
Dans la partie droite de la fenetre, il faut alors choisir le type de tache (ponctuelle ou longue), puis saisir
un intitule, une date de debut, une date de fin, et choisir une categorie. En validant la saisie, la nouvelle tache s'insere dans la liste a gauche. 
Pour supprimer une tache, il suffit de cliquer sur le bouton supprimer correspondant, dans le champ le plus a droite.
\\
\subsection*{Tache ponctuelle}
Pour une tache ponctuelle, lorsque celle ci a été éffectuée, l'utilisateur doit alors saisir 100(\%) dans le champs le plus a gauche, "Effectuée". Ainsi, la tache disparait de la liste car elle n'est plus a faire.
Si une tache ponctuelle n'a pas été effectuée au lendemain de la date de fin, alors elle est en retard et apparait en rouge dans la liste.
\\

\subsection*{Tache longue}
Pour une tache longue, differentes étapes sont a effectuer et peuvent amener a un retard. Ainsi, l'utilisateur pourra saisir l'avancement de sa tache en pourcentage, dans le champ "Effectuée".
Lorsque l'utilisateur saisi 100 dans ce champ, la tache est entièrement effectuée et disparait. Si l'avancement est inferieur a 100\%, alors le calcul du retard se fait par tranche de quart de la durée totale, et ainsi la tache apparait en rouge si:
\begin{itemize}
\item l'avancement est inferieur a 25\% au quart de la durée totale
\item l'avancement est inferieur a 50\% a la moitié de la durée totale
\item l'avancement est inferieur a 75\% aux trois-quarts de la durée totale
\item l'avancement est inferieur a 100\% au lendemain de la date de fin de la tache\\
\end{itemize}
Dans la liste, les taches longues se differencient des autres en apparaissant en italique.

\end{document}
